\chapter{Introduction}

Mathematics uses a precise language, and that's not a language we learn unless we specifically take classes catered to either mathematicians or philosophers.  The language emphasizes exact precision and logic.  The definitions, theorems, proofs: they all are meticulously crafted to leave no opening for ambiguity.  Everything needs to be precisely defined.  If you are unused to this language, check out Appendix~\ref{ch:notation} for some of the shorthand we use.

Humans don't naturally think this way.  Only by reading and writing a lot of technical math, feeling stupid for a long time, and turning your brain to a pile of mush can you train your mind to work at this higher level of precision.  That way of thinking will not just benefit your mathematics skills, but your all-around ability to precisely define concepts and problems in any situation.  Precision and logical progression can help most difficulties.

The other day, I was playing a grammar game, where I had to decide if a particular sentence was correct or not.  The sentence was, "He stole the blue woman's purse".  With my mathematical thinking, I got the question wrong, because technically, nothing is wrong with having a "blue woman".  Precisely, the sentence was grammatically correct.  Just much less probable than the person intending "He stole the woman's blue purse".  That's the difference between mathematical thinking and conventional thinking.

A joke runs about a biologist, a physicist, and a mathematician traveling through Ireland when they spot a flock of white sheep.  The biologist concludes that "Sheep in Ireland are white".  The physicist concludes that "Some sheep in Ireland are white".  The mathematician concludes that "At least one side of some sheep in Ireland are white".

But at the end of the day, all of this abstract language terms serve to convey a beautiful wonderland that just \textit{can't} be adequately described through any other wording.  It's like trying to describe blue to a blind person or a statue to someone who lives in two dimensions.  Each person needs to get to the idea for themselves through hard work.

Yet most sources assume that since precise definitions are ``necessary'', they are also ``sufficient''.  That means precise math talk is the only thing you get to go on. But I don't believe precise definitions are sufficient. So here, I will do my best to give you pictures, everyday examples, philosophical ponderings, jokes, and all the insights I've gained from my hours of intellectual masochism.

Enjoy.

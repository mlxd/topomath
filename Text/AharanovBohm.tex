\section{Spin in Magnetic Field}
The non-relativistic Hamiltonian for a free particle in a magnetic field
\begin{equation}
  \mathcal{H}=-\frac{\hbar^2}{2m}\left(\grad-i\frac{q}{c\hbar}\mathbf{A} \right)^2
\end{equation}
contains a dependence on a gauge field $\omega$, since the gradient of the any $\omega$ can be added to the the vector potential $\mathbf{A}$ without changing the physical magnetic field
\begin{equation}
  \mathbf{B}=\grad \times \left(\mathbf{A}+\grad \omega \right)=\grad \times \mathbf{A}
\end{equation}

Suppose that $\Psi (\mathbf{r})$ is a solution to the $\mathbf{A}=0$ non-magnetic situation.  Then
\begin{equation}
  \Psi^A(\mathbf{r}) = e^{i\frac{q}{c\hbar} \int_{\mathbf{r}_0}^{\mathbf{r}} \mathbf{A}(\mathbf{r}^{\prime}) \cdot d\mathbf{r}^{\prime}   }
  \Psi(\mathbf{r})
\end{equation}
is a solution to the general vector potential $\mathbf{A}\neq 0$ Hamiltonian.
For ease, let $i\frac{q}{c\hbar} \int_{\mathbf{r}_0}^{\mathbf{r}} \mathbf{A}(\mathbf{r}^{\prime}) \cdot d\mathbf{r}^{\prime} = \phi$.
\begin{subequations}
  \begin{equation}
    \grad^2  -\left(\frac{q}{c\hbar}\right)^2\mathbf{A}^2
    -i\frac{q}{c\hbar} \mathbf{A} \grad
    -i \frac{q}{c \hbar} \left((\grad \mathbf{A}) + \mathbf{A}\grad \right)
   \end{equation}
   \begin{equation}
     \grad^2  -\left(\frac{q}{c\hbar}\right)^2\mathbf{A}^2
     -2 i\frac{q}{c\hbar} \mathbf{A} \grad
     -i \frac{q}{c \hbar} \left(\grad \mathbf{A}\right)
   \end{equation}

   Calculating a $\grad$
   \begin{equation}
     \left(i \frac{q}{c \hbar} \right) \mathbf{A}
     e^{\phi} \Psi (\mathbf{r}) +
     e^{\phi} \grad \Psi (\mathbf{r})
   \end{equation}
   Taking just the $\grad^2$ term
   \begin{equation}
     e^{\phi} \left(
    \grad^2 \Psi(\mathbf{r})
     +
    2\frac{i q}{c \hbar} \mathbf{A} \grad \Psi(\mathbf{r})
     -
     \left(\frac{q}{c\hbar}\right)^2 \mathbf{A}^2  \Psi(\mathbf{r})
     +
     \frac{i q}{c \hbar}
      \Psi(\mathbf{r}) \grad \mathbf{A} \right)
   \end{equation}
\end{subequations}

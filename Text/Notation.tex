\chapter{Notation}\label{ch:notation}
I understand not everyone will have the same amount of mathematical training, and though I do use pictures and handwaving, I don't shy away from formal definitions and notation.  Precision is necessary.  So here, I will cover some of the symbols and phrases that I usually take from granted people know.

\begin{definition}[Exists]
  $\exists$ is shorthand for "there exists".  It does not make any constraints on the object other than existance.
\end{definition}

\begin{definition}[For all]
  $\forall$ is shorthand for "for all".  Often used in definitions or constraints.
\end{definition}

\begin{definition}[In]
  For members of a set, I will use $\in$ to denote membership.  For example, apples $\in$ fruits.
\end{definition}

\begin{definition}[Implies]
  I will use $\rightarrow$ is we can logically deduce one thing from another.
\end{definition}

\subsection{Notation for Special Sets}
\begin{enumerate}
  \item $\mathbb{R}$ the Real line, for example, 1, $\pi$, $\sqrt{2}$, $569543/3874329$, and $.3234898957786...$
  \item $\mathbb{Z}$ the Integers: $-1, 0, 1, 2,...$
  \item $\mathbb{Q}$ The Rationals: any number that can be written as the ratio of two integers $p/q : p,q \in \mathbb{Z}$.
  \item $S^1$ The points on on the surface of a circle.
  \item $S^2$ The points on the surface of a sphere.
  \item $S^n$ The points on the surface on an n-sphere.  I hope you get the trend now.
\end{enumerate}
